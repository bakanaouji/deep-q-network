\documentclass{jarticle}
\usepackage{fancybox}
\usepackage{amsmath,amssymb,amsfonts}
\usepackage[dvipdfmx]{graphicx}
\usepackage{wrapfig}%文字の画像回り込み
\usepackage{comment}
\usepackage{multicol}
\usepackage{txfonts,here}
\usepackage{algorithm} %algorithmとalgorithmic環境を利用するのに必要.
\usepackage{algcompatible}
\algblockdefx{FORP}{ENDFORP}[1]%
  {\textbf{for}#1 \textbf{do in parallel}}%
  {\textbf{end for}}
\usepackage{bm}
\usepackage{longtable}
\usepackage{multirow}
\usepackage{colortbl}
\usepackage{subfigure}
\usepackage{listings}
\usepackage[top=30truemm,bottom=30truemm,left=25truemm,right=25truemm]{geometry}
\usepackage{url}
% \newcommand{\argmax}{\mathop{\rm arg~max}\limits}
% \newcommand{\argmin}{\mathop{\rm arg~min}\limits}
%%%% ↓algotithmic の \REQUIRE と \ENSURE の表記を変更する
\renewcommand{\algorithmicrequire}{\textbf{Input:}}
\renewcommand{\algorithmicensure}{\textbf{Output:}}
%%%% ↑algotithmic の \REQUIRE と \ENSURE の表記を変更する

\begin{document}
\begin{algorithm}[tb]
\caption{Deep Q-learning with experience replay}
\label{alg:dqn}                          
\begin{algorithmic}[1]   
\REQUIRE $r_{\mathrm{learn}},n_{\mathrm{rmem}},\epsilon_{\mathrm{final}},l_{\mathrm{expl}},f_{\mathrm{learn}},f_{\mathrm{update}},\gamma,n_{\mathrm{batch}}, M$
\STATE Replay Memory$~D$を初期化
\STATE Q-Network$~Q$をランダムな重み$\theta$で初期化
\STATE Target network$~Q^-$を重み$\theta^-=\theta$で初期化
\FOR{$\mathrm{episode}=1,\cdots,M$}
\STATE $t=1$
\WHILE{not $done$}
\STATE ε-greedyに従って行動$a_t$を選択
\STATE $\epsilon=\max{(\epsilon_{\mathrm{final}},\epsilon-\frac{1-\epsilon_{\mathrm{final}}}{l_{\mathrm{expl}}})}$→εを線形減少
\STATE 行動$a_t$を実行し,報酬$r_t$と次の画面$x_{t+1}$と$done$を観測
\STATE 前処理して次の状態$s_{t+1}$を生成
\STATE $D$に$(s_t,a_t,r_t,s_{t+1},done)$を追加,$|D|>n_{\mathrm{nmem}}$なら古いものを削除する.
\IF{$t>n_{\mathrm{rpstart}}$}
\IF{$(t-1)\%f_{\mathrm{learn}}=0$}
\STATE $D$からランダムに$(s_j,a_j,r_j,s_{j+1},done)$を$n_{\mathrm{batch}}$個の履歴をサンプル
\STATE $y_j=
\begin{cases}
r_j & (done) \\
r_j+\gamma\max_{a^{\prime}}Q^-(s_{j+1},a^{\prime};\theta^-) & (\mbox{otherwise})
\end{cases}$
\STATE $\theta$を$(y_j-Q(s_j,a_j;\theta))^2$の勾配方向に学習率$r_{\mathrm{learn}}$で更新(勾配計算の後の更新の際,$r_j$は$[-1,1]$にクリップされる)
\ENDIF
\IF{$(t-1)\%f_{\mathrm{update}}=0$}
\STATE $Q^-=Q$
\ENDIF
\ENDIF
\STATE $t=t+1$
\ENDWHILE
\ENDFOR
\end{algorithmic}
\end{algorithm}

\begin{algorithm}[tb]
\caption{Environment wrapping Atari Games}
\label{alg:env}                          
\begin{algorithmic}[1]   
\REQUIRE $n_{\mathrm{batch}},n_{\mathrm{rmem}},l_{\mathrm{history}},f_{\mathrm{tnupdate}},\gamma,r_{\mathrm{learn}},m_{\mathrm{gradient}},m_{\mathrm{sgradient}},g_{\mathrm{min}},\epsilon_{\mathrm{first}},\epsilon_{\mathrm{final}},l_{\mathrm{expl}},n_{\mathrm{rpstart}},l_{\mathrm{nomax}}$
\STATE Replay Memory$~D$を初期化
\STATE Q-Network$~Q$をランダムな重み$\theta$で初期化
\STATE Target network$~Q^-$を重み$\theta^-=\theta$で初期化
\FOR{$\mathrm{episode}=1,\cdots,M$}
\STATE $T\sim U(1,l_{\mathrm{nomax}})$
\FOR{$t^{\prime}=1,\cdots,T$}
\STATE $a_t^{\prime}=(\mbox{do nothing})$の実行→初期状態の生成
\ENDFOR
\STATE $t=1$
\WHILE{not $done$}
\IF{$t\leq n_{\mathrm{rpstart}}$}
\STATE $a_t$をランダムに決定→Replay Memoryの確保
\ELSE
\IF{$(t-1)\%l_{\mathrm{history}}=0$}
\STATE ε-greedyに従って行動$a_t$を選択
\ELSE
\STATE $a_t=a_{t-1}$
\ENDIF
\STATE $\epsilon=\max{(\epsilon_{\mathrm{final}},\epsilon-\frac{\epsilon_{\mathrm{first}}-\epsilon_{\mathrm{final}}}{l_{\mathrm{expl}}})}$→εを線形減少
\ENDIF
\STATE 行動$a_t$を実行し,報酬$r_t$と次の画面$x_{t+1}$と$done$を観測
\STATE 前処理して次の状態$s_{t+1}$を生成
\STATE $D$に$(s_t,a_t,r_t,s_{t+1},done)$を追加,$|D|>n_{\mathrm{nmem}}$なら古いものを削除する.
\IF{$t>n_{\mathrm{rpstart}}$}
\IF{$(t-1)\%l_{\mathrm{history}}=0$}
\STATE $D$からランダムに$(s_j,a_j,r_j,s_{j+1},done)$を$n_{\mathrm{batch}}$個の履歴をサンプル
\STATE $y_j=
\begin{cases}
r_j & (done) \\
r_j+\gamma\max_{a^{\prime}}Q^-(s_{j+1},a^{\prime};\theta^-) & (\mbox{otherwise})
\end{cases}$
\STATE $\theta$を$(y_j-Q(s_j,a_j;\theta))^2$の勾配方向にRMSProp$(r_{\mathrm{learn}},m_{\mathrm{gradient}},m_{\mathrm{sgradient}},g_{\mathrm{min}})$を用いて更新(勾配計算の後の更新の際,$r_j$は$[-1,1]$にクリップされる)
\ENDIF
\IF{$(t-1)\%(f_{\mathrm{tnupdate}}\times l_{\mathrm{history}})=0$}
\STATE $Q^-=Q$
\ENDIF
\ENDIF
\STATE $t=t+1$
\ENDWHILE
\ENDFOR
\end{algorithmic}
\end{algorithm}

\end{document}
